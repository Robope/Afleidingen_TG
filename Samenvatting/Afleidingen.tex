% !TeX spellcheck = nl_NL
\documentclass[a4paper,kul]{kulakarticle} %options: kul or kulak (default)

\usepackage[utf8]{inputenc}
\usepackage[dutch]{babel}
\usepackage[T1]{fontenc}
\date{Academiejaar 2021 -- 2022}
\address{
	Industriële Ingenieurswetenschappen \\
	Trillingen \& Golven \\
	Maarten Vanierschot}
\title{Afleidingen}
\author{\href{https://github.com/debber1}{Robbe Decapmaker}}
\usepackage{hyperref}
\usepackage{graphicx}
\usepackage{amsmath, amssymb, amsthm}
\usepackage{siunitx}
\usepackage{flafter} 
\usepackage{pdfpages}
\usepackage{pgfplots}
\usepackage{caption}
\usepackage{subcaption}

\begin{document}

\maketitle
\section*{Dedicated to Rebecca}
\section*{Inleiding}

De afleidingen voor trillingen en golven. \href{https://github.com/debber1/Afleidingen_TG}{De source code is te vinden op github.}\\
https://github.com/debber1/Afleidingen\_TG\\
\newline
Dit document is \textbf{niet} alles wat je moet kennen van theorie voor het examen. Ik ben niet verantwoordelijk voor jouw resultaat op jouw examen. Het is jouw verantwoordelijkheid om ook nog de andere onderdelen van deze cursus op een degelijke manier te verwerken. Het is namelijk zo dat dit document enkel wiskundige afleidingen bevat. Onderwerpen zoals polarisatie en radar zijn ook te kennen.\\

\section*{Contributors}
\href{https://github.com/debber1}{Robbe Decapmaker}\\
\href{https://github.com/sydon1}{Sodir Yuksel} \\
\href{https://github.com/ItsAlphie}{Jonathan Valgaeren}\\
\href{https://github.com/Robope}{Robbe Willaert}
\newpage
\section{Afleiding 1}

\textbf{Afleiding van de uitdrukking voor de verplaatsing van een ongedempte trilling aan de hand van de bewegings-vergelijking}\\
\begin{figure}[htbp]
	\centering
	\includegraphics[width=0.4\linewidth]{MassaVeer}
	\caption[Massa veer systeem]{Massa veer systeem}
	\label{fig:massaveer}
\end{figure}
\begin{figure}[htbp]
	\centering
	\includegraphics[width=0.7\linewidth]{Harmonische_Oscillator}
	\caption[Harmonische Oscillator]{Harmonische Oscillator}
	\label{fig:harmonischeoscilator}als dit het punt nie over brengt, weet ik het ook niet meer :p
\end{figure}

We stellen eerst de tweede wet van Newton op voor het blokje M uit figuur \ref{fig:massaveer}.
\begin{equation*}
	\sum \vec{F} = m\vec{a}
\end{equation*}
We kijken nu enkel naar de x-component en brengen de kracht van de veer in rekening:
\begin{equation*}
	-kx = m\frac{d^2x}{dt^2}
\end{equation*}
We hebben ook de versnelling geschreven als de tweede afgeleide van de verplaatsing. Door alle termen naar het linker lid te verplaatsen krijgen we volgende differentiaal vergelijking: 
\begin{equation}
	m\frac{d^2x}{dt^2} + kx = 0
	\label{eq:DVGLTrilling}
\end{equation}
\newpage
Deze differentiaal vergelijking moeten we oplossen aan de hand van beginvoorwaarden. Deze voorwaarden verkrijgen we experimenteel. Bij het experiment noteren we de uitwijking tegenover de tijd. Hierdoor verkrijgen we figuur \ref{fig:harmonischeoscilator}.
Wiskundig vertaalt dit zich tot: 
\begin{equation}
	x(t) = A cos(\omega t + \varphi)
	\label{eq:trilling}
\end{equation}
We kunnen vergelijking \ref{eq:trilling} een eerste keer afleiden om de snelheid van het blokje te verkrijgen: 
\begin{equation}
	v(t) = \frac{dx(t)}{dt} = -\omega Asin(\omega t + \varphi)
	\label{eq:trillingsnelheid}
\end{equation}
Door vergelijking \ref{eq:trilling} nogmaals af te leiden, kunnen we ook de versnelling van het blokje verkrijgen:
\begin{equation}
	a(t) = \frac{d^2x(t)}{dt^2} = -\omega^2 Acos(\omega t + \varphi)
	\label{eq:trillingversnelling}
\end{equation}
Nu kunnen we vergelijkingen \ref{eq:trillingversnelling} en \ref{eq:trilling} invullen in vergelijking \ref{eq:DVGLTrilling}:
\begin{equation}
	-\omega^2 mAcos(\omega t + \varphi) + k A cos(\omega t + \varphi) = 0
	\label{eq:Bewegingsvergelijking}
\end{equation}
Vergelijking \ref{eq:Bewegingsvergelijking} is nu een oplossing voor differentiaal vergelijking \ref{eq:DVGLTrilling} als aan volgende voorwaarde voldaan is:
\begin{align*}
	-\omega^2 mAcos(\omega t + \varphi) + k A cos(\omega t + \varphi) & = 0\\
	(\frac{k}{m}-\omega^2)A cos(\omega t + \varphi) & = 0\\
	\frac{k}{m}-\omega^2 & = 0\\
	\frac{k}{m} & = \omega^2
\end{align*}


\newpage
\section{Afleiding 2}
\textbf{Afleiding van de potentiële en kinetische energie van een ongedempte trilling in functie van de plaats}
\\

Er zijn 2 vormen van energie aanwezig in een massa-veer-systeem. We hebben de potentiële energie in de veer en de kinetische energie van het blokje.
\begin{align*}
	E = & \frac{1}{2}kx^2 + \frac{1}{2}mv^2\\
	E(x) = & \frac{1}{2}k A^2cos^2(\omega t +\varphi) + \frac{1}{2}m \omega^2 A^2 sin^2(\omega t + \varphi)\\
	E(x) = & \frac{1}{2}k A^2cos^2(\omega t +\varphi) + \frac{1}{2}m \frac{k}{m} A^2 sin^2(\omega t + \varphi)\\
	E(x) = & \frac{1}{2} k A^2 (cos^2(\omega t +\varphi) + sin^2(\omega t + \varphi))\\
	E(x) = & \frac{1}{2} k A^2
\end{align*} 
We zien nu duidelijk dat de hoeveelheid energie niet afhankelijk is van de uitwijking van de massa. Met andere woorden: de energie blijft constant in het systeem (figuur \ref{fig:energiebalans}).
\begin{figure}[htbp]
	\centering
	\includegraphics[width=0.7\linewidth]{Energie_Balans}
	\caption[Energie balans]{Energie balans}
	\label{fig:energiebalans}
\end{figure}

\newpage

\section{Afleiding 3}
\textbf{Snelheid in functie van de afstand}
\begin{equation*}
	\frac{dv}{dt} = \frac{dv}{dx}\cdot\frac{dx}{dt}
\end{equation*}
We kunnen vergelijkingen \ref{eq:trillingsnelheid} en \ref{eq:trillingversnelling} invullen:
\begin{equation*}
	-A\omega^2cos(\omega t + \varphi) = \frac{dv}{dx}\cdot (-A\omega sin(\omega t +\varphi))
\end{equation*}
Na schrappen en herschikken:
\begin{equation*}
	\frac{dv}{dx} = \frac{\omega cos(\omega t + \varphi)}{sin(\omega t + \varphi)}
\end{equation*}
Sinus substitutie:
\begin{align*}
	\frac{dv}{dx} =& \frac{\omega cos(\omega t + \varphi)}{\sqrt{1-cos^2(\omega t + \varphi)}}\\
	\frac{dv}{dx} =& \frac{\pm \omega \frac{x}{A}}{\sqrt{1-\frac{x^2}{A^2}}}
\end{align*}
Nu integratie van beide leden:
\begin{equation*}
	\int dv = \int \frac{\pm \omega \frac{x}{A}}{\sqrt{1-\frac{x^2}{A^2}}} dx
\end{equation*}
De snelheid in functie van de afstand is dus:
\begin{equation}
	\label{eq:snelheid_positie}
	v(x) = \omega A\sqrt{1-\frac{x^2}{A^2}}
\end{equation}
Uit vergelijking \ref{eq:snelheid_positie} kunnen we nu ook afleiden wat de uitdrukking is voor $v_{\text{max}}$. We weten dat $v_{\text{max}}$ zich zal voordoen bij het evenwichtspunt oftewel als x = 0.
\begin{align*}
	v(x) =& \omega A\sqrt{1-\frac{x^2}{A^2}}\\
	v_{\text{max}} =& \omega A\sqrt{1-\frac{0^2}{A^2}}\\
	v_{\text{max}} =& \omega A\sqrt{1}\\
	v_{\text{max}} =& \omega A
\end{align*}
\newpage
\section{Afleiding 4}
\textbf{Afleiding van de beweging van een pendulum aan de hand van de bewegingsvergelijking}
\begin{figure}[htbp]
	\centering
	\includegraphics[width=0.5\linewidth]{Pendulum}
	\caption[Pendulum]{Pendulum}
	\label{fig:pendulum}
\end{figure} \\
De drijvende kracht van de pendulum is gelijk aan:
\begin{equation*}
	F = -mg sin(\theta)
\end{equation*}
Hiermee kunnen we de bewegingsvergelijking opstellen:
\begin{equation*}
	m\frac{d^2x}{dt^2} = -mgsin(\theta)
\end{equation*}
We kunnen stellen dat $sin(\theta) \approx \theta$ voor een kleine $\theta$:
\begin{equation*}
	m\frac{d^2x}{dt^2} = -mg\theta
\end{equation*}
We kunnen ook stellen dat $x = l\theta$:
\begin{equation*}
	ml\frac{d^2\theta}{dt^2} = -mg\theta
\end{equation*}
We kunnen dit nu herschrijven naar:
\begin{equation*}
	\frac{d^2\theta}{dt^2} +\frac{g}{l}\theta = 0
\end{equation*}
Dit geeft ons een differentiaalvergelijking. Een oplossing ziet er als volgt uit:
\begin{equation*}
	\theta (t) = \theta_0 cos(\omega t +\varphi)
\end{equation*}
Hierbij zien we dat $\omega = \sqrt{\frac{g}{l}}$.
\newpage
\section{Afleiding 5}
\textbf{Gedempte harmonische trilling}\\
\begin{figure}[htbp]
	\centering
	\includegraphics[width=0.7\linewidth]{DempingVeer}
	\caption[Gedempt massa veer systeem]{Gedempt massa-veer-systeem}
	\label{fig:dempingveer}
\end{figure}
We beginnen met de bewegingsvergelijking op te stellen:
\begin{align*}
	m \vec{a} =& \vec{F}_{veer} + \vec{F}_{demper}\\
	m \frac{d^2x}{dt^2} = & -kx - bv\\
	m \frac{d^2x}{dt^2} = & -kx - b\frac{dx}{dt}\\
	0 = & m \frac{d^2x}{dt^2} + b\frac{dx}{dt} + kx
\end{align*}
We maken gebruik van de formule van Euler $e^{i\varphi}=cos(\varphi) +isin(\varphi)$ om tot een oplossing van de differentiaalvergelijking te komen:
\begin{equation*}
	\widetilde{x}(t) = Ae^{(\gamma +i\omega)t+i\varphi}
\end{equation*}
Het reëel deel is dus:
\begin{equation*}
	x(t) = Re(\widetilde{x}(t)) = Ae^{\gamma t}cos(\omega t + \varphi)
\end{equation*}
We kunnen deze nu invullen in de bewegingsvergelijking:
\begin{equation}
	\label{eq:complexbeweging}
	m(\gamma +i\omega)^2\widetilde{x}(t) + b(\gamma +i\omega)e^{(\gamma +i\omega)t+i\varphi}+ke^{(\gamma +i\omega)t+i\varphi} = 0
\end{equation}
Na herordenen en het delen door m, krijgen we:
\begin{equation*}
	\gamma^2+2i\gamma\omega-\omega^2+\frac{b}{m}\gamma+i\frac{\omega b}{m}+\frac{k}{m}=0
\end{equation*}
We kijken nu naar het imaginair deel:
\begin{align*}
	2i\gamma \omega +i\frac{\omega b}{m}&=0\\
	2\gamma + \frac{b}{m} = 0\\
	-\frac{b}{2m} = \gamma
\end{align*}
We vullen nu deze waarde van $\gamma$ in in het reëel deel van vergelijking \ref{eq:complexbeweging}:
\begin{align*}
	\gamma^2-\omega^2+\frac{b}{m}\gamma+\frac{k}{m} &=0\\
	(-\frac{b}{2m})^2-\omega^2+\frac{b}{m}(-\frac{b}{2m})+\frac{k}{m} &=0\\ 
	\frac{k}{m}-\frac{b^2}{4m^2}&=\omega^2\\
	\sqrt{\frac{k}{m}-\frac{b^2}{4m^2}}&=\omega
\end{align*}
Hiermee hebben we nu de natuurlijke frequentie van het systeem gevonden. We kunnen tot slot ook nog stellen dat:
\begin{equation*}
	x(t) = Re(\widetilde{x}(t)) = Ae^{-\frac{b}{2m} t}cos(\omega t + \varphi)
\end{equation*}
We kunnen nu drie gevallen onderscheiden: 
\begin{figure}[htbp]
	\centering
	\includegraphics[width=0.5\linewidth]{Ondergedempt}
	\caption[Ondergedempte trilling]{Ondergedempte trilling ($b^2 << 4mk$)}
	\label{fig:ondergedempt}
\end{figure}
\begin{figure}[htbp]
	\centering
	\includegraphics[width=0.5\linewidth]{Over_Kritisch_gedempt}
	\caption[Kritisch en over gedempte trilling]{Kritisch ($b^2 = 4mk$) en over gedempte ($b^2 >> 4mk$) trilling}
	\label{fig:overkritischgedempt}
\end{figure}


\newpage
\section{Afleiding 6}
\textbf{Gedwongen trillingen}\\
We stellen eerst de bewegingsvergelijking op:
\begin{equation}
	m\frac{d^2\widetilde{x}(t)}{dt^2}+b\frac{d\widetilde{x}(t)}{dt}+k\widetilde{x}(t) = F_0cos(\omega t)
	\label{eq:gedwongenbewegings}
\end{equation}
Een voorstel voor de algemene oplossing van differentiaalvergelijking \ref{eq:gedwongenbewegings} ziet er als volgt uit:
\begin{equation*}
	\widetilde{x}(t) = Ae^{i(\omega t +\varphi)}
\end{equation*}
waarbij $\widetilde{F} = F_0e^{i\omega t}$.
Dit kunnen we nu invullen in vergelijking \ref{eq:gedwongenbewegings}:
\begin{align*}
	-m\omega^2\widetilde{x}(t)+i\omega b\widetilde{x}(t)+k\widetilde{x}(t) & =F_0e^{i\omega t}\\
	-m\omega^2A^{i\omega t}e^{i\varphi}+i\omega bA^{i\omega t}e^{i\varphi}+kA^{i\omega t}e^{i\varphi} & =F_0e^{i\omega t}\\
	-m\omega^2A+i\omega b+k & =F_0e^{-i\varphi} 
\end{align*}
We kunnen dit opsplitsen in een imaginair en reëel deel:
\begin{description}
	\item[Imaginair:]$\omega bA = -F_0sin(\varphi)$
	\item[Reëel:] $-m\omega^2 A +kA = F_0cos(\varphi)$
\end{description}
Als we nu het imaginair deel delen door het reëel deel verkrijgen we voor $\varphi$:
\begin{align*}
	-\frac{\omega b}{k-m\omega^2} &= tan(\varphi)\\
	bgtan(\frac{-\omega b}{k-m\omega^2}) & =\varphi
\end{align*}
Om de amplitude te bepalen kijken we opnieuw naar het imaginair deel:
\begin{align*}
	\omega bA &= -F_0sin(\varphi)\\
	\omega bA &= -F_0\frac{tan(\varphi)}{\sqrt{1+tan^2(\varphi)}}\\
	\omega bA &= -F_0\frac{tan(bgtan(\frac{-\omega b}{k-m\omega^2}))}{\sqrt{1+tan^2(bgtan(\frac{-\omega b}{k-m\omega^2}))}}\\
	\omega bA &= -F_0\frac{\frac{-\omega b}{k-m\omega^2}}{\sqrt{1+\frac{\omega^2 b^2}{(k-m\omega^2)^2}}}\\
	\omega bA &= -F_0\frac{\frac{-\omega b}{k-m\omega^2}}{\sqrt{\frac{(k-m\omega^2)^2+\omega^2 b^2}{(k-m\omega^2)^2}}}\\
	A &= \frac{\frac{F_0}{k-m\omega^2}}{\sqrt{\frac{(k-m\omega^2)^2+\omega^2 b^2}{(k-m\omega^2)^2}}}\\
	A &= \frac{\frac{F_0}{m}}{\sqrt{(\frac{k}{m}-\omega^2)^2+\frac{\omega^2b^2}{m^2}}}\\
	A &= \frac{\frac{F_0}{m}}{\sqrt{(\omega_0^2-\omega^2)^2+\frac{\omega^2b^2}{m^2}}}	
\end{align*}
met $\omega_0$ de natuurlijke frequentie van het systeem en $\omega$ de gedwongen frequentie.
\newpage
\section{Afleiding 7}
\textbf{Golf vergelijking}\\
We nemen aan da $F_r$ en $F_l$ uit figuur \ref{fig:touwgolf} gelijk zijn aan elkaar en aan de spanning in het touw $\sigma A$. we stellen ook dat de massa voldoet aan $dm=\rho dsA$.\\
\begin{figure}[htbp]
	\centering
	\includegraphics[width=0.7\linewidth]{"touw_golf"}
	\caption[Touw]{Touw}
	\label{fig:touwgolf}
\end{figure}\\
We starten met het opstellen van de wet van Newton:
\begin{align*}
	\sum\vec{F}&=m\vec{a}\\
	F_rsin(\alpha_r) - F_lsin(\alpha_l) & = dma_y\\
	\sigma A(sin(\alpha_r)-sin(\alpha_l)) & = dm\frac{\partial^2y}{\partial t^2}\\
	\sigma(\frac{dy}{dx}|_r-\frac{dy}{dx}|_l) & = \rho ds \frac{\partial^2y}{\partial t^2}
\end{align*}
We stellen dat $\alpha$ klein is, en dus kunnen we zeggen dat $sin(\alpha)\approx tan(\alpha) = \frac{dx}{dy}$. We nemen ook aan dat $ds \approx dx$.
\begin{align*}
	\sigma\frac{\frac{dy}{dx}|_r-\frac{dy}{dx}|_l}{dx} & = \rho \frac{\partial^2y}{\partial t^2}\\
	\sigma\frac{\partial^2y}{\partial x^2} & = \rho \frac{\partial^2y}{\partial t^2}\\
	\frac{\partial^2y}{\partial t^2} & = \frac{\sigma}{\rho}\frac{\partial^2y}{\partial x^2}\\
	\frac{\partial^2y}{\partial t^2} & = v^2\frac{\partial^2y}{\partial x^2}
\end{align*}
Hiermee zijn we de 1D golfvergelijking bekomen.
\newpage
\section{Afleiding 8}
\textbf{Harmonische golven}\\
Een voorstel voor de 1D golf vergelijking ziet er als volgt uit:
\begin{equation*}
	y(x,t) = y_msin(kx \pm \omega t +\varphi)
\end{equation*}
We kunnen deze oplossing invullen in de golfvergelijking:
\begin{equation*}
	-\omega^2y_msin(kx \pm \omega t +\varphi)=v^2k^2y_msin(kx \pm \omega t +\varphi)
\end{equation*}
Als we dit vereenvoudigen, krijgen we de voorwaarden waaraan een golf moet voldoen.
\begin{align*}
	v^2 & = \frac{\omega^2}{k^2}\\
	v & = \pm\frac{\omega}{k}
\end{align*}
We kunnen nu een analyse doen op de golfvergelijking in het tijdsdomein:\\
\begin{figure}[h]
	\centering
	\includegraphics[width=0.7\linewidth]{tijdsdomein_analyse}
	\caption[Tijdsdomein analyse]{Tijdsdomein analyse}
	\label{fig:tijdsdomein}
\end{figure}\\
We starten opnieuw met de algemene oplossing, maar dan op vaste plaats $x_1$:
\begin{align}
	y(x_1,t) &= y_msin(kx_1 \pm \omega t +\varphi)\\
	\label{eq:tijdsanalyse1}
	y(x_1,t) &= y_msin(\pm \omega t +\varphi_1)
\end{align}
met $\varphi_1 = kx_1 +\varphi$. We kunnen dit opnieuw doen voor $x_2$:
\begin{align}
	y(x_2,t) &= y_msin(kx_2 \pm \omega t +\varphi)\\
	\label{eq:tijdsanalyse2}
	y(x_2,t) &= y_msin(\pm \omega t +\varphi_2)
\end{align}
met $\varphi_2 = kx_2 +\varphi$. Uit vergelijkingen \ref{eq:tijdsanalyse1} en \ref{eq:tijdsanalyse2} stellen we vast dat er enkel een fase verschuiving is. We vermelden ook nog dat oscillaties in fase zijn als $2\pi = k\lambda$.
\newpage
We doen ook de analyse doen op de golfvergelijking in de ruimte:\\
\begin{figure}[h]
	\centering
	\includegraphics[width=0.7\linewidth]{ruimtelijke_analyse}
	\caption[Ruimtelijke analyse]{Ruimtelijke analyse}
	\label{fig:ruimtelijkeanalyse}
\end{figure}\\
We starten opnieuw met de algemene oplossing, maar dan op vast tijdstip $t_1$:
\begin{align}
	y(x,t_1) &= y_msin(kx \pm \omega t_1 +\varphi)\\
	\label{eq:ruimteanalyse1}
	y(x,t_1) &= y_msin(kx +\varphi_1)
\end{align}
met $\varphi_1 = \varphi \pm \omega t_1$. We stellen dit opnieuw op voor $t_2$
\begin{align}
	y(x,t_2) &= y_msin(kx \pm \omega t_2 +\varphi)\\
	y(x,t_2) &= y_msin(kx \underbrace{\pm \omega t_1 +\varphi}_{\varphi_1}\pm\Delta t)\\
	\label{eq:ruimteanalyse2}
	y(x,t_2) &= y_msin(kx +\varphi_2)
\end{align}
met $\varphi_2 = \varphi_1\pm\Delta t$.
Uit de waarden van $\varphi_1$ en $\varphi_2$ kunnen we de voortplantingsrichting van de golf bepalen: 
\begin{center}
	\begin{tabular}{|c|c|}
		\hline
		$\varphi_1$ < $\varphi_2$& Golf naar rechts \\
		\hline
		$\varphi_1$ > $\varphi_2$& Golf naar links \\
		\hline
	\end{tabular} 
\end{center}
We kunnen ook iets zeggen over de voortplantingssnelheid $v$.
\begin{align*}
 v = \frac{dx_{max}}{dt} = \frac{d}{dt}(kx_{max} \pm \omega t +\varphi) &= 0\\
 k\frac{dx_{max}}{dt}\pm \omega & = 0\\
 \pm \frac{\omega}{k} & = v\\
 \pm\frac{2\pi f}{\frac{2\pi}{\lambda}} & = v\\
 \pm \lambda f & = v
\end{align*}
\newpage
\section{Afleiding 9}
\textbf{Superpositie}\\
We zeggen dat de algemene oplossing $y(x,t)$ van de golfvergelijking kan uitgedrukt worden door 2 nieuwe functies:
\begin{equation*}
	y(x,t) = y_1(x,t)+y_2(x,t)
\end{equation*}
We vullen dit opnieuw in in de golfvergelijking:
\begin{align*}
	\frac{\partial^2 y(x,t)}{\partial t^2} & = v^2 \frac{\partial^2 y(x,t)}{\partial x^2}\\
	\frac{\partial^2 y_1(x,t)+y_2(x,t)}{\partial t^2} & = v^2 \frac{\partial^2 y_1(x,t)+y_2(x,t)}{\partial x^2}\\
\end{align*}
We gebruiken de lineariteit van de afgeleide:
\begin{equation*}
	\frac{\partial^2 y_1(x,t)}{\partial t^2} +\frac{\partial^2 y_2(x,t)}{\partial t^2}  = v^2(\frac{\partial^2 y_1(x,t)}{\partial x^2}+\frac{\partial^2 y_2(x,t)}{\partial x^2})
\end{equation*}
\newpage
\section{Afleiding 10}
\textbf{Interferentie}\\
Interferentie doet zich voor als er meerdere verschillende golven aanwezig zijn in 1 medium. We kunnen dit voorstellen door volgende 2 functies (zie ook figuur \ref{fig:interferentie}):
\begin{align*}
	y_1 & = y_msin(kx\pm\omega t)\\
	y_2 & = y_msin(kx\pm\omega t +\phi)
\end{align*}
\begin{figure}[h]
	\centering
	\includegraphics[width=0.5\linewidth]{Interferentie}
	\caption[Interferentie]{Golven $y_1$ en $y_2$ die zullen interfereren}
	\label{fig:interferentie}
\end{figure}\\
We bepalen nu hoe $y_1+y_2$ er wiskundig uit zal zien aan de hand van de \href{https://nl.wikipedia.org/wiki/Lijst_van_goniometrische_gelijkheden#Som-naar-product-identiteiten_(regels_van_Simpson)}{regels van Simpson}:
\begin{align*}
	y & = y_1+y_2\\
	y & = y_msin(kx\pm\omega t) y_m+sin(kx\pm\omega t +\phi)\\
	y & = \underbrace{2y_mcos(\frac{\phi}{2})}_{A = f(\phi)}sin(kx\pm\omega t+\frac{\phi}{2})
\end{align*} \\
De 2 uitersten van dit fenomeen krijgen een naam: destructieve- en constructieve-interferentie. Bij destructieve zullen de golven elkaar tegen werken zodat de beweging stopt (zie figuur \ref{fig:destructief}). Constructieve interferentie zal ervoor zorgen dat de resulterende golf groter wordt (zie figuur \ref{fig:constructief}).
\begin{figure}[h]
	\centering
	\begin{subfigure}{.4\textwidth}
		\centering
		\includegraphics[width=1\linewidth]{constructieve_interferentie}
		\caption{Constructieve interferentie}
		\label{fig:constructief}
	\end{subfigure}%
	\begin{subfigure}{.4\textwidth}
		\centering
		\includegraphics[width=1\linewidth]{destructieve_interferentie}
		\caption{Destructieve interferentie}
		\label{fig:destructief}
	\end{subfigure}
	\caption{Twee extreme gevallen van interferentie.}
	\label{fig:omegaVZ}
\end{figure}\\
\newpage
\section{Afleiding 11}
\textbf{Staande golven}\\
Staande golven doen zich vaak voor bij muziek instrumenten zoals de gitaar. We een staande golf als volgt wiskundig voorstellen:
\begin{align*}
	y(x,t) &=y_msin(kx-\omega t)+y_msin(kx+\omega t)\\
	y(x,t) &=\underbrace{2y_msin(kx)}_{A(x)}cos(\omega t)
\end{align*}
Hierbij gebruiken we wederom de \href{https://nl.wikipedia.org/wiki/Lijst_van_goniometrische_gelijkheden#Som-naar-product-identiteiten_(regels_van_Simpson)}{regels van Simpson}. Omdat de uiteinden van de snaar (met lengte $L$) vast gehouden worden, kunnen we volgende randvoorwaarden stellen:
\begin{align*}
	y(0,t)  &= 0 \Rightarrow sin(0)=0\\
	y(L,t) &= 0 \Rightarrow sin(kL)=0
\end{align*}
We kunnen dus algemeen zeggen dat $k\cdot L$ altijd gelijk moet zijn aan een veelvoud van $\pi$ oftewel:
\begin{align*}
	kL & = m\pi (m\in\mathbb{N})\\
	\frac{2\pi}{\lambda}L &=m\pi\\
	\lambda & = \frac{2L}{m}
\end{align*}
\textbf{Eerste harmonische $(m = 1)$}\\
\begin{figure}[h]
	\centering
	\includegraphics[width=0.7\linewidth]{Eerste_harm}
	\caption[Eerste harmonische]{Eerste harmonische}
	\label{fig:eersteharm}
\end{figure}\\
Hierbij is $m = 1$ dus moet $\lambda = 2L$ en de frequentie is dan $f = \frac{v}{\lambda} = \frac{v}{2L}$.\\
\newpage
\textbf{Tweede harmonische $(m = 2)$}\\
\begin{figure}[h]
	\centering
	\includegraphics[width=0.6\linewidth]{Tweede_harm}
	\caption[Tweede harmonische]{Tweede harmonische}
	\label{fig:tweedeharm}
\end{figure}\\
Hierbij is $m = 2$ dus moet $\lambda = L$ en de frequentie is dan $f = \frac{v}{\lambda} = \frac{v}{L}$.\\
\textbf{Derde harmonische $(m = 3)$}\\
\begin{figure}[h]
	\centering
	\includegraphics[width=0.6\linewidth]{Derde_harm}
	\caption[Derde harmonische]{Derde harmonische}
	\label{fig:Derdeharm}
\end{figure}\\
Hierbij is $m = 3$ dus moet $\lambda = \frac{2}{3}L$ en de frequentie is dan $f = \frac{v}{\lambda} = \frac{3v}{2L}$.\\
\newpage
\section{Afleiding 12}
\textbf{Energie transport in een golf}\\
\begin{figure}[h]
	\centering
	\includegraphics[width=0.7\linewidth]{EnergieGolf}
	\caption[Energie transport]{Energie transport}
	\label{fig:energietransport}
\end{figure}\\
De energie in een massa-veer-systeem is:
\begin{equation*}
	E = \frac{1}{2}kA^2 = dE = \frac{1}{2}\omega^2dmA^2
\end{equation*}
We gebruiken nu het feit dat $\omega = \sqrt{\frac{k}{m}} $ en $\omega = \pi f$:
\begin{equation*}
	dE = 2\pi^2f^2dmA^2
\end{equation*}
We kunnen $dm$ herschrijven als $dm = \rho Sdx = \rho Sdtv$:
\begin{align*}
	dE &= 2\pi^2f^2A^2\rho Svdt\\
	\frac{dE}{dt} &= 2\pi^2f^2A^2\rho Sv = P
\end{align*}
We bekomen het vermogen van de golf. We kunnen hieruit vervolgens de intensiteit $I$ halen:
\begin{equation*}
	I = \frac{P}{S} = 2\pi^2f^2A^2\rho v
\end{equation*}
\newpage
\section{Afleiding 13}
\textbf{Eigenschappen geluid}\\
\begin{figure}[h]
	\centering
	\includegraphics[width=0.7\linewidth]{EigenschappenGeluid}
	\caption[Zuiger]{Drukgolf in een zuiger}
	\label{fig:drukgolfzuiger}
\end{figure}\\
Met $v'$ de snelheid van de zuiger en $v$ de geluidsnelheid kunnen we stellen dat:
\begin{align*}
	V_0 &=Sv\Delta t\\
	\Delta V &= -Sv'\Delta t
\end{align*}
De kracht is dan:
\begin{align*}
	F_{\text{net}} &= (P_0+\Delta P)S-P_0S\\
	F_{\text{net}}&= \Delta PS
\end{align*}
Als we rekening houden met het feit dat $dm=\rho V_0$ kunnen we met behulp van impuls berekenen dat:
\begin{align*}
	F_{\text{net}} \Delta t &=dmv'\\
	F_{\text{net}}\Delta t &= \rho v\Delta tSv'\\
	\Delta PS &= \rho vSv'\\
	\Delta P &= \rho vv'
\end{align*}
\newpage
\section{Afleiding 14}
\textbf{Wiskundige beschrijving van geluid}\\
\begin{figure}[h]
	\centering
	\includegraphics[width=0.7\linewidth]{WiskundigeBeschrijving}
	\caption[Drukgolven]{Drukgolven in een buis}
	\label{fig:drukgolvenbuis}
\end{figure}\\
\begin{figure}[h]
	\centering
	\includegraphics[width=0.5\linewidth]{Drukinbuis}
	\caption[Drukgolven]{Drukgolven in een buis}
	\label{fig:drukgolvenbuisgrafiek}
\end{figure}\\

Met de compressie- of bulk-modulus $B = \frac{\Delta P}{\frac{\Delta V}{V_0}}$ kunnen we zeggen dat:
\begin{align*}
	\Delta P &= -B \frac{\Delta V}{dv}\\
	\Delta P &= -B \frac{\Delta DS}{\Delta xS}\\
	\Delta P &= -B \frac{\partial D}{\partial x}
\end{align*}
We geven een voorstel voor $D$: $D = \Delta sin(kx\pm \omega t)$
\begin{align*}
	\Delta P &= -BAkcos(kx\pm \omega t)\\
	&= -\rho v^2Akcos(kx\pm \omega t)\\
	&= -\rho v\omega Acos(kx\pm \omega t)\\
	&= \underbrace{-2\pi\rho vfA}_{\Delta P_m} cos(kx\pm \omega t)
\end{align*}
\newpage
\section{Afleiding 15}
\textbf{Trillende lucht kolommen}\\
We modelleren de verplaatsing van lucht als:
\begin{equation}
	\label{eq:Verplaatsing}
	D = Asin(kx\pm\omega t)
\end{equation}
De druk kunnen we schrijven als:
\begin{align}
	\Delta P &= -\Delta P_M cos(kx\pm\omega t)\\
	\label{eq:Druk}
	&= -\Delta P_M sin(kx\pm\omega t-\frac{\pi}{2})
\end{align}
Uit vergelijking \ref{eq:Verplaatsing} en \ref{eq:Druk} kunnen we afleiden dat de druk een faseverschil van $\frac{\pi}{2}$ heeft tegenover de verplaatsing zoals te zien is in figuur \ref{fig:faseverschildrukverplaatsing}. 
\begin{figure}[h]
	\centering
	\includegraphics[width=0.7\linewidth]{FaseVerschilDrukVerplaatsing}
	\caption[Fase vershil druk verplaatsing]{Fase verschil tussen druk en verplaatsing}
	\label{fig:faseverschildrukverplaatsing}
\end{figure}
We zien hier duidelijk dat de druk na ijlt op de verplaatsing. 
\newpage
\section{Afleiding 16}
\textbf{Staande golven in een open buis}\\
We veronderstellen een open buis met lengte L. Als nevenvoorwaarde stellen we dat de druk aan de uiterste kanten van de buis gelijk is aan de atmosferische druk $P_{atm}$. We kijken eerst naar de drukverdeling in de buis:
\begin{align*}
	\Delta P_S (x,t) &= \Delta P_m sin(kx+\omega t) + \Delta P_m sin(kx-\omega t)\\
	&= 2\Delta P_m sin(kx)cos(\omega t)
\end{align*}
Hierbij gebruiken we de \href{https://nl.wikipedia.org/wiki/Lijst_van_goniometrische_gelijkheden#Som-naar-product-identiteiten_(regels_van_Simpson)}{regels van Simpson}. Vervolgens kijken we naar de verplaatsing in de buis:
\begin{align*}
	D_S(x,t)&=Asin(kx+\omega t+\frac{\pi}{2})+Asin(kx-\omega t+\frac{\pi}{2})\\
	&= 2Asin(kx+\frac{\pi}{2})cos(\omega t)
\end{align*}
Hierbij gebruiken we nogmaals de \href{https://nl.wikipedia.org/wiki/Lijst_van_goniometrische_gelijkheden#Som-naar-product-identiteiten_(regels_van_Simpson)}{regels van Simpson}.\\
Enkele oplossingen zien er als volgt uit:\\

\textbf{Eerste harmonische}\\
Hierbij is $\lambda = 2L$ en $f=\frac{v}{2L}$.

\begin{figure}[h]
	\centering
	\begin{subfigure}{.5\textwidth}
		\centering
		\includegraphics[width=1\linewidth]{OpenBuisEersteDruk}
		\caption{Drukverdeling}
		\label{fig:EersteBuisDruk}
	\end{subfigure}%
	\begin{subfigure}{.5\textwidth}
		\centering
		\includegraphics[width=1\linewidth]{OpenBuisEersteVerplaatsing}
		\caption{Verplaatsing}
		\label{fig:EersteBuisVerplaatsing}
	\end{subfigure}
	\caption{Druk en verplaatsing voor de eerste harmonische in een open buis}
	\label{fig:OpenBuisEerste}
\end{figure}
\newpage
\textbf{Tweede harmonische}\\
Hierbij is $\lambda = L$ en $f=\frac{v}{L}$.

\begin{figure}[!h]
	\centering
	\begin{subfigure}{.5\textwidth}
		\centering
		\includegraphics[width=1\linewidth]{OpenBuisTweedeDruk}
		\caption{Drukverdeling}
		\label{fig:TweedeBuisDruk}
	\end{subfigure}%
	\begin{subfigure}{.5\textwidth}
		\centering
		\includegraphics[width=1\linewidth]{OpenBuisTweedeVerplaatsing}
		\caption{Verplaatsing}
		\label{fig:TweedeBuisVerplaatsing}
	\end{subfigure}
	\caption{Druk en verplaatsing voor de tweede harmonische in een open buis}
	\label{fig:OpenBuisTweede}
\end{figure}

\textbf{Derde harmonische}\\
Hierbij is $\lambda = \frac{2L}{3}$ en $f=\frac{3v}{2L}$.

\begin{figure}[!h]
	\centering
	\begin{subfigure}{.5\textwidth}
		\centering
		\includegraphics[width=1\linewidth]{OpenBuisDerdeDruk}
		\caption{Drukverdeling}
		\label{fig:DerdeBuisDruk}
	\end{subfigure}%
	\begin{subfigure}{.5\textwidth}
		\centering
		\includegraphics[width=1\linewidth]{OpenBuisDerdeVerplaatsing}
		\caption{Verplaatsing}
		\label{fig:DerdeBuisVerplaatsing}
	\end{subfigure}
	\caption{Druk en verplaatsing voor de derde harmonische in een open buis}
	\label{fig:OpenBuisDerde}
\end{figure}
\newpage
\section{Afleiding 17}
\textbf{Staande golven in een half open buis}\\
\begin{figure}[!h]
	\centering
	\begin{subfigure}{.5\textwidth}
		\centering
		\includegraphics[width=1\linewidth]{HalfOpenBuisEersteDruk}
		\caption{Drukverdeling}
		\label{fig:EersteHalfBuisDruk}
	\end{subfigure}%
	\begin{subfigure}{.5\textwidth}
		\centering
		\includegraphics[width=1\linewidth]{HalfOpenBuisEersteSnelheid}
		\caption{Snelheid}
		\label{fig:EersteHalfBuisSnelheid}
	\end{subfigure}
	\caption{Druk en snelheid voor de eerste harmonische in een half open buis}
	\label{fig:HalfOpenBuisEerste}
\end{figure}\\
We kunnen opnieuw enkele randvoorwaarden stellen:
\begin{equation*}
	D_s(L,t)=0\Rightarrow kl+\frac{\pi}{2} = \pi
\end{equation*}
Dit wil zeggen dat:
\begin{equation*}
	k=\frac{\pi}{2L}=\frac{2\pi}{\lambda}\Rightarrow \lambda=4L \wedge f=\frac{v}{4L}
\end{equation*}
We kunnen ook een tweede nevenvoorwaarde opstellen:
\begin{equation*}
	kL+\frac{\pi}{2}=2\pi
\end{equation*}
Dit wil zeggen dat:
\begin{equation*}
	k=\frac{3\pi}{2L}=\frac{2\pi}{\lambda}\Rightarrow \lambda=\frac{4L}{3} \wedge f=\frac{3v}{4L}
\end{equation*}
\newpage
\section{Afleiding 18}
\textbf{Interferentie van geluidsgolven}\\
\begin{figure}[h]
	\centering
	\includegraphics[width=0.3\linewidth,angle=90]{InterferentieSpeakers}
	\caption[Interferentie]{Interferentie bij speakers}
	\label{fig:interferentiespeakers}
\end{figure}
\begin{figure}[h]
	\centering
	\includegraphics[width=0.5\linewidth]{SpeakersWiskundig}
	\caption[Interferentie wiskundig]{Interferentie wiskundig}
	\label{fig:speakerswiskundig}
\end{figure}\\
We kunnen de druk beschrijven als:
\begin{align*}
	\Delta P_1 &= \Delta P_msin(kL_1\pm\omega t)\\
	\Delta P_2 &= \Delta P_msin(kL_2\pm\omega t)
\end{align*}
Door beide op te tellen en de regels van Simpson toe te passen krijgen we:
\begin{equation*}
	\Delta P = 2\Delta P_msin(k\frac{L_1+L_2}{2}\pm\omega t)cos(k\frac{L_1-L_2}{2})
\end{equation*}
We onderscheiden 2 soorten interferentie zoals te zien in figuur \ref{fig:interferentiespeakers}. Constructieve interferentie (groen) en destructieve interferentie (rood).

Bij constructieve interferentie is $\frac{k\Delta L}{2}=0,\pi,2\pi,\ldots$ Dit wil zeggen dat $\Delta L$ moet voldoen aan $\Delta L = 0,\lambda,2\lambda,\ldots$. Destructieve interferentie doet zich voor als $\frac{k\Delta L}{2}=\frac{\pi}{2},\frac{3\pi}{2},\ldots$. $\Delta L$ moet hier dus voldoen aan $\Delta L = \frac{\lambda}{2},\frac{3\lambda}{2},\ldots$.

\newpage
\section{Afleiding 19}
\textbf{Kloppen van een golf}\\
Als 2 golven frequenties hebben die zeer dicht bij elkaar liggen, krijgen we kloppingen. We kunnen de druk van beide golven voorstellen als:
\begin{align*}
	D_1&=Asin(2\pi f_1t)\\
	D_2&=Asin(2\pi f_2t)
\end{align*}
Door beide op te tellen en de regels van Simpson toe te passen krijgen we:
\begin{equation*}
	D=2Acos(2\pi\frac{f_1-f_2}{2}t)sin(2\pi\frac{f_1+f_2}{2}t)
\end{equation*}
Hierbij vermelden we ook nog dat als $f_1\approx f_2\Rightarrow\Delta\omega=\frac{f_1-f_2}{2}<<1$.


\newpage
\section{Afleiding 20}
\textbf{Doppler effect}\\
Het doppler effect doet zich voor als de bron en waarnemer van een geluid in beweging zijn tegenover elkaar. We analyseren nu het geval waarin de bron beweegt: als de bron naar rechts beweegt,  zal een waarnemer die zich links bevindt een lagere frequentie waarnemen. Een waarnemer die rechts staat, zal daarentegen een hogere frequentie waarnemen. De nieuwe frequentie heet $\lambda'$. Wiskundig kunnen we $\lambda'$ bepalen met volgende redenering:
\begin{equation*}
	\lambda'=\lambda\pm v_{bron}T=\lambda\pm v_{bron}\frac{\lambda}{v_s}=\lambda(1\pm\frac{v_{bron}}{v_s})
\end{equation*}
De nieuwe frequentie is dan $f'=\frac{f}{1\pm\frac{v_{bron}}{v_s}}$.
\begin{center}
	\begin{tabular}{|c|c|}
		\hline
		Bron naar je toe  & f$\nearrow$ \\
		\hline
		Bron van je weg & f $\searrow$\\
		\hline
	\end{tabular}
	
\end{center}
\newpage
\section{Afleiding 21}
\textbf{Basiswetten voor elektromagnetische golven}\\
\paragraph{Wet van Ampère}
\begin{equation*}
	\oint\vec{B}\vec{dl}=\mu I_{\text{lus}}+\mu_0\epsilon_0\frac{d\Phi_E}{dt}
\end{equation*}
waarbij $\Phi_E$ de elektrische flux is:
\begin{figure}[h]
	\centering
	\includegraphics[width=0.5\linewidth]{ElektrischVeld}
	\caption[Elektrisch veld]{Elektrisch veld}
	\label{fig:elektrischveld}
\end{figure}
\begin{align*}
	\Phi_E &= E_{\perp}A\\
	&= Ecos(\theta)A\\
	&= \vec{E}\cdot\vec{A}\quad\text{(Scalair product)}
\end{align*}
\paragraph{Wetten van maxwell}
\begin{align}
	\oint\vec{E}\vec{dA}&=\frac{Q}{\epsilon_0}\\
	\oint\vec{B}\vec{dA}&=0\\
	\label{eq:faraday}
	\oint\vec{E}\vec{dl}&=-\frac{\partial\Phi_B}{\partial t}\\
	\label{eq:maxwell}
	\oint\vec{B}\vec{dl}&=\mu\epsilon_0\frac{\partial\Phi_E}{\partial t}
\end{align}
waarbij vergelijking \ref{eq:faraday} en \ref{eq:maxwell} invloed hebben op elektromagnetische golven.
\newpage
\section{Afleiding 22}
\textbf{Generatie EM golven}\\
Men kan elektromagnetische golven genereren door twee platen naast elkaar te zetten met een wisselspanningsbron zoals in figuur \ref{fig:emgolf1}:
\begin{figure}[h]
	\centering
	\includegraphics[width=0.3\linewidth]{EMGolf1}
	\caption[Generatie]{Generatie EM golf}
	\label{fig:emgolf1}
\end{figure}\\
Als de spanningsbron van polariteit veranderd, veranderen ook de velden van richting zoals in figuur \ref{fig:emgolf2}. Hier kunnen we tevens ook zien hoe de golf zich voortplant in de ruimte: na verloop van tijd zullen de golven een lineair karakter krijgen. 
\begin{figure}[h]
	\centering
	\includegraphics[width=0.7\linewidth]{EMGolf2}
	\caption[Generatie en voortplanting]{Generatie en voortplanting EM golf}
	\label{fig:emgolf2}
\end{figure}
\newpage
\section{Afleiding 23}
\textbf{Elektromagnetische velden}\\
Eerst enkele eigenschappen van elektromagnetische golven
\begin{enumerate}
	\item $\vec{B}$ en $\vec{E}$ staan loodrecht op de voortplantingsrichting (transversale golven)
	\item $\vec{B}$ staat loodrecht op $\vec{E}$
	\item $\vec{B} \times\vec{E}$ geeft de voortplantingsrichting van de golf
	\item $\vec{B}$ en $\vec{E}$ zijn harmonische golven met $\Delta\Phi = 0$
\end{enumerate}
\begin{figure}[h]
	\centering
	\includegraphics[width=0.7\linewidth]{An-electromagnetic-wave}
	\caption[Elektromagnetische golf]{Elektromagnetische golf}
	\label{fig:an-electromagnetic-wave}
\end{figure}
In figuur \ref{fig:an-electromagnetic-wave} kunnen we stellen dat:
\begin{align*}
	\vec{E} &= E_msin(kx-\omega t)\vec{j}\\
	\vec{B} &= B_msin(kx-\omega t)\vec{k}
\end{align*}
\newpage
\section{Afleiding 24}
\textbf{Opstellen van de golfvergelijking van een elektromagnetische golf}\\
\textbf{Wet van Faraday}\\
\begin{figure}[h]
	\centering
	\begin{subfigure}{.5\textwidth}
		\centering
		\includegraphics[width=1\linewidth]{FaradayGolf}
		\caption[Elektrische golf]{Elektrische golf}
		\label{fig:faradaygolf}
	\end{subfigure}%
	\begin{subfigure}{.5\textwidth}
		\centering
		\includegraphics[width=0.6\linewidth]{InzoomFaraday}
		\caption[Inzoom Faraday]{Uitvergroting}
		\label{fig:inzoomfaraday}
	\end{subfigure}
	\caption{Wet van Faraday}
	\label{fig:InductieFaraday}
\end{figure}\\

We berekenen de verandering in het elektrische flux van een golf. We definiëren hiervoor een gesloten kring zoals te zien is in figuur \ref{fig:faradaygolf}. We nemen dx zo klein mogelijk, zodat we de golf op dit kleine interval als lineair kunnen aannemen. We zien deze linearisering in figuur \ref{fig:inzoomfaraday}. Hierdoor wordt het effectief elektrisch veld gereduceerd tot de twee dikke lijnen op de linker en rechter zijde van de rechthoek. Deze zijn respectievelijk gelijk aan:
\begin{align*}
	\vec{E_L} &= \vec{E}-\frac{\partial\vec{E}}{\partial x}\frac{dx}{2}\\
	\vec{E_R} &= \vec{E}+\frac{\partial\vec{E}}{\partial x}\frac{dx}{2}	
\end{align*}
Door de wet van Faraday $-\frac{\partial\Phi_B}{\partial t}= \oint\vec{E}\cdot\vec{dS}$ toe te passen op alle 4 de zijden van de rechthoek:
\begin{equation*}
	\oint\vec{E}\cdot\vec{dS} = \underbrace{(E+\frac{\partial E}{\partial x}\frac{dx}{2})h}_1 + \underbrace{0}_2 -\underbrace{(E-\frac{\partial E}{\partial x}\frac{dx}{2})h}_3 + \underbrace{0}_4
\end{equation*}
Hierbij zijn de scalaire producten voor zijde 2 en 4 gelijk aan nul omdat de vectoren loodrecht op elkaar staan. We kunnen deze uitdrukking verder vereenvoudigen tot (met de wet van Faraday):
\begin{align}
	 \frac{\partial E}{\partial x}hdx &= -\frac{\partial B}{\partial t}hdx\\
	 \label{eq:inductieFaraday}
	 \frac{\partial E}{\partial x} &= -\frac{\partial B}{\partial t}
\end{align}
\newpage
\textbf{Wet van Maxwell}\\
\begin{figure}[h]
	\centering
	\begin{subfigure}{.5\textwidth}
		\centering
		\includegraphics[width=1\linewidth]{MaxwellGolf}
		\caption[Golf Maxwell]{Magnetische golf}
		\label{fig:maxwellgolf}
	\end{subfigure}%
	\begin{subfigure}{.5\textwidth}
		\centering
		\includegraphics[width=0.6\linewidth]{InzoomMaxwell}
		\caption[Inzoom maxwell]{Uitvergroting}
		\label{fig:inzoommaxwell}
	\end{subfigure}
	\caption{Wet van Maxwell}
	\label{fig:InductieMaxwell}
\end{figure}
We maken opnieuw dezelfde redenering als voor de afleiding met de wet van Faraday, we merken wel op dat we nu met een magnetisch veld werken en dat we gebruik maken van de wet van Maxwell ($\mu_0\epsilon_0\frac{\partial \Phi_E}{\partial t}=\oint\vec{B}\cdot\vec{dS}$). De twee dikke lijnen kunnen we hierdoor definiëren als:
\begin{align*}
	\vec{B_L} &= \vec{B}-\frac{\partial\vec{B}}{\partial x}\frac{dx}{2}\\
	\vec{B_R} &= \vec{B}+\frac{\partial\vec{B}}{\partial x}\frac{dx}{2}	
\end{align*}
We passen de wet van Maxwell toe op de randen van de gesloten kring:
\begin{equation*}
	\oint\vec{B}\cdot\vec{dS} = \underbrace{(B-\frac{\partial B}{\partial x}\frac{dx}{2})h}_1 + \underbrace{0}_2 -\underbrace{(B+\frac{\partial B}{\partial x}\frac{dx}{2})h}_3 + \underbrace{0}_4
\end{equation*}
Hierbij zijn de scalaire producten voor zijde 2 en 4 gelijk aan nul omdat de vectoren loodrecht op elkaar staan. We kunnen deze uitdrukking verder vereenvoudigen tot (met de wet van Maxwell):
\begin{align}
	-\frac{\partial B}{\partial x}hdx &= \mu_0\epsilon_0\frac{\partial E}{\partial t}hdx\\
	\label{eq:inductieMaxwell}
	-\frac{\partial B}{\partial x} &= \mu_0\epsilon_0\frac{\partial E}{\partial t}
\end{align}
\newpage
\textbf{Opstellen van de golfvergelijking met de inductie wetten}\\
We werken nu verder met vergelijking \ref{eq:inductieFaraday} en \ref{eq:inductieMaxwell}. We leiden eerst vergelijking \ref{eq:inductieFaraday} af naar de tijd om volgende uitdrukking te krijgen:
\begin{equation}
	\label{eq:DubbelFaraday}
	\frac{\partial^2E}{\partial t\partial x} = \frac{\partial^2B}{\partial t^2}
\end{equation}
We leiden nu ook vergelijking \ref{eq:inductieMaxwell} af maar dit keer naar de ruimte:
\begin{equation}
	\label{eq:DubbelMaxwell}
	-\frac{\partial^2B}{\partial x^2} = \mu_0\epsilon_0\frac{\partial^2E}{\partial t\partial x}
\end{equation}
We kunnen nu vergelijking \ref{eq:DubbelFaraday} en \ref{eq:DubbelMaxwell} samenbrengen om de golfvergelijking op te stellen:
\begin{equation*}
	\frac{\partial^2B}{\partial t^2}=\frac{1}{\mu_0\epsilon_0}\frac{\partial^2B}{\partial t^2}
\end{equation*}
Hierbij vermelden we nog dat $c^2=\frac{1}{\mu_0\epsilon_0}$.
\newpage
\section{Afleiding 25}
\textbf{Energie van een elektromagnetische golf}\\
We definiëren de energie dichtheid van het elektrisch en magnetisch veld als volgt: 
\begin{align*}
	U_E&=\frac{1}{2}\epsilon_0E^2\\
	U_B&=\frac{1}{2}\frac{B^2}{\mu_0}
\end{align*}
We stellen dat de totale energie dichtheid van een elektromagnetische golf gegeven wordt door de som van deze dichtheden:
\begin{equation*}
	U_{\text{tot}} = \frac{1}{2}\epsilon_0E^2+\frac{1}{2}\frac{B^2}{\mu_0}
\end{equation*}
We kunnen nu de individuele bijdrage van elk veld bepalen door gebruik te maken van de wet van Faraday, meer bepaald vergelijking \ref{eq:inductieFaraday}:
\begin{align*}
	\frac{\partial}{\partial x}(E_msin(kx-\omega t)) &=-\frac{\partial}{\partial t}(B_msin(kx-\omega t))\\
	E_mkcos(kx-\omega t) &= B_m\omega cos(kx-\omega t)\\
	\frac{E_m}{B_m} &= \frac{\omega}{k} = c
\end{align*}
Hieruit volgt dat:
\begin{equation*}
	E_m=cB_m
\end{equation*}
Als we de verkregen relatie tussen E en B in de totale energiedichtheid invullen krijgen we:
\begin{align*}
	U_{\text{tot}} &= \frac{1}{2}\epsilon_0E^2+\frac{1}{2}\frac{B^2}{\mu_0}\\
	U_{\text{tot}} &= \frac{1}{2}\epsilon_0E^2+\frac{1}{2}\frac{E^2}{\mu_0c^2}\\
	U_{\text{tot}} &= \underbrace{\frac{1}{2}\epsilon_0E^2}_{\text{50\%}}+\underbrace{\frac{1}{2}\epsilon_0E^2}_{\text{50\%}}
\end{align*}
De totale energie valt ook makkelijk te berekenen als:
\begin{equation*}
	U_{\text{tot}}=\epsilon_0E^2=\epsilon_0cEB = \frac{EB}{\mu_0}
\end{equation*}
We kijken nu ook naar de relatie tussen energie dichtheid en vermogen. We kijken naar de hoeveelheid energie die een elektromagnetische golf kan afzetten op een volume:
\begin{equation*}
	dU = \epsilon_0E^2dV=\epsilon_0E^2Acdt
\end{equation*}
We gaan nu van energie over naar vermogen door deze afzetting te bekijken over een klein tijdsinterval dt:
\begin{equation*}
	\frac{dU}{dt}=\frac{\epsilon_0E^2Acdt}{dt}= \epsilon_0E^2Ac
\end{equation*}
Het vermogen per oppervlakte wordt dan:
\begin{equation*}
	S = \frac{\epsilon_0E^2Ac}{A} = \epsilon_0E^2c = \frac{EB}{\mu_0}
\end{equation*}
We kijken nu naar de Poynting vector $\vec{S}$:
\begin{equation*}
	\vec{S} = \frac{\vec{E}\times\vec{B}}{\mu_0}
\end{equation*}
\newpage
Tot slot definiëren we ook nog de intensiteit, het gemiddelde van het vermogen per oppervlakte:
\begin{align*}
	S&= \frac{EB}{\mu_0}\\
	&= \frac{E^2}{\mu_0c}\\
	&= \frac{1}{\mu_0c}E^2_msin^2(kx-\omega t)\\
	&=\frac{E^2_m}{\mu_0c}(\frac{1}{2}-\frac{1}{2}cos(2(kx-\omega t)))
\end{align*} 
We nemen nu het gemiddelde van $\vec{S}$:
\begin{equation*}
	\vec{S}=\frac{1}{T}\int_{0}^{T}Sdt=\frac{E^2_m}{2\mu_0c}=\frac{E_{\text{RMS}}B_{\text{RMS}}}{\mu_0}
\end{equation*}
\newpage
\section{Afleiding 26}
\textbf{Bewijs de gelijkheid van de lichtsnelheid via eenheden}\\
We zullen bewijzen dat de volgende gelijkheid klopt met betrekking tot de eenheden:
\begin{equation*}
	c = \frac{1}{\sqrt{\mu_0\epsilon_0}}
\end{equation*}
We vullen de eenheden in van alle constanten en werken dan verder:
\begin{align*}
	\frac{m}{s} &= \frac{1}{\sqrt{(\frac{N}{A^2})(\frac{C^2}{N\cdot m^2})}}\\
	\frac{m}{s} &= \frac{1}{\sqrt{(\frac{s^2}{C^2})(\frac{C^2}{ m^2})}}\\
	\frac{m}{s} &= \frac{1}{\sqrt{(\frac{s^2}{m^2})}}\\
	\frac{m}{s} &= \frac{1}{\frac{s}{m}}\\
	\frac{m}{s} &= \frac{m}{s}	
\end{align*}
\newpage
\section{Afleiding 27}
\textbf{Afleiding van de werking van een sferische spiegel voor verschillende posities van het object}\\
We bespreken eerst de situatie waarbij een object tussen C en f staat, zoals te zien is in figuur \ref{fig:objecttussen}.
\begin{figure}[h]
	\centering
	\includegraphics[width=0.7\linewidth]{ObjectTussen}
	\caption[Object tussen C en f]{Object tussen C en f}
	\label{fig:objecttussen}
\end{figure}\\
We zoeken een waarde voor $d_I$, daarvoor gebruiken we de geometrie van de driehoeken O'OA en IAI'.  We weten dus dat: 
\begin{align*}
	\Delta O'OA&\Rightarrow tan(\theta) = \frac{|OO'|}{d_O}\\
	\Delta IAI'&\Rightarrow tan(\theta) = \frac{|II'|}{d_I}
\end{align*}
Hieruit leiden we af dat de versterking $m$ gelijk is aan :
\begin{equation*}
	\frac{h_O}{d_O} = \frac{h_I}{d_I} = |m|
\end{equation*}
Om nu de afstand $d_I$ te vinden kunnen de twee congruente driehoeken die in het geel aangeduid staan op figuur \ref{fig:objecttussen} gebruiken. Voor deze driehoeken geldt dat: 
\begin{equation*}
	\frac{|OO'|}{|Of|}=\frac{|AB'|}{|Af|} \approx\frac{|AB'|}{f} \Rightarrow \frac{h_O}{d_0-f} = \frac{h_I}{f}
\end{equation*}
Dit kunnen we verder uitwerken tot:
\begin{equation*}
	\frac{h_O}{h_I} = \frac{d_O-f}{f}=\frac{d_O}{d_I}\Rightarrow\frac{1}{f}-\frac{1}{d_O} = \frac{1}{d_I}
\end{equation*}
\newpage
We bespreken nu de situatie waarbij het object tussen f en de spiegel staat, zoals te zien is in figuur \ref{fig:objectspiegelf}.
\begin{figure}[h]
	\centering
	\includegraphics[width=0.7\linewidth]{ObjectSpiegelF}
	\caption[Object tussen f en spiegel]{Object tussen f en spiegel}
	\label{fig:objectspiegelf}
\end{figure}\\
We proberen opnieuw om de afstand $d_I$ te bepalen aan de hand van driehoeken.
\begin{align*}
	\Delta fBA&\Rightarrow tan(\theta)=\frac{|AB|}{f}\approx\frac{h_O}{f}\\
	\Delta fII'&\Rightarrow tan(\theta)= \frac{h_I}{f+d_I}
\end{align*}
Hieruit volgt dat:
\begin{equation*}
	\frac{h_0}{h_I}=\frac{f}{f+d_I}
\end{equation*}
We kijken nog naar 2 andere driehoeken
\begin{align*}
	\Delta CII'&\Rightarrow tan(\theta)= \frac{h_I}{2f+d_I}\\
	\Delta COO'&\Rightarrow tan(\theta)= \frac{h_O}{2f-d_O}
\end{align*}
Hieruit volgt dat:
\begin{equation*}
	\frac{h_0}{h_I}=\frac{2f-d_O}{2f+d_I}
\end{equation*}
We kunnen dus stellen dat:
\begin{align*}
	\frac{f}{f+d_I}&=\frac{2f-d_O}{2f+d_I}\\
	\frac{1}{d_O} &= \frac{1}{f}+\frac{1}{d_I}
\end{align*}
\newpage
We bespreken nu de situatie waarbij het object ver weg staat, zoals te zien is in figuur \ref{fig:rondespiegelvoorwerpver}.
\begin{figure}[h]
	\centering
	\includegraphics[width=0.7\linewidth]{RondeSpiegelVoorwerpVer}
	\caption[Spiegel ver]{Object staat ver van de spiegel}
	\label{fig:rondespiegelvoorwerpver}
\end{figure}
We proberen opnieuw om de afstand $d_I$ te bepalen aan de hand van driehoeken.
\begin{align*}
	\Delta OO'A&\Rightarrow tan(\theta)=\frac{h_O}{d_O}\\
	\Delta II'A&\Rightarrow tan(\theta)= \frac{h_I}{d_I}
\end{align*}
Hieruit volgt dat:
\begin{equation*}
	\frac{h_O}{h_I}=\frac{d_O}{d_I}
\end{equation*}
We kijken nog naar 2 andere driehoeken
\begin{align*}
	\Delta OO'f&\Rightarrow tan(\theta)=\frac{h_I}{d_O-f}\\
	\Delta fAB&\Rightarrow tan(\theta)= \frac{h_I}{f}
\end{align*}
Hieruit volgt dat:
\begin{equation*}
	\frac{h_O}{h_I}=\frac{d_O-f}{f}
\end{equation*}
We kunnen dus stellen dat:
\begin{align*}
	\frac{d_O}{d_I}&=\frac{d_O-f}{f}\\
	\frac{1}{f}&= \frac{1}{d_I}+\frac{1}{d_O}
\end{align*}
Alle van bovenstaande inzichten moeten ook toegepast kunnen worden op een convexe spiegel. 
\newpage
\section{Afleiding 28}
\textbf{Afleiding van Snellius}\\
\begin{figure}[!h]
	\centering
	\includegraphics[width=0.7\linewidth]{Snellius}
	\caption[Snellius]{Wet van Snellius}
	\label{fig:snellius}
\end{figure}\\
Als een golf van medium veranderd zoals te zien is in figuur \ref{fig:snellius}, zal er een verandering in voorplantingsrichting optreden. we maken de aanname dat de brekingsindex van het tweede medium kleiner is dan dat van het eerste medium in figuur \ref{fig:snellius}. We kunnen dus stellen dat:
\begin{align*}
	n_2&<n_1\\
	\frac{c}{v_2}&<\frac{c}{v_1}\\
	v_2&>v_1
\end{align*}
We stellen ook dat de afstand tussen de golffronten in medium 1 gelijk is aan: $\lambda_1$. We kunnen dus volgende redenering maken (omdat de frequentie van de golf niet veranderd):
\begin{align*}
	v_1&=\lambda_1f\\
	v_2&=\lambda_2f
\end{align*}
We kunnen deze vergelijkingen delen door elkaar:
\begin{equation}
	\label{eq:verbandSnellius}
	\frac{v_1}{v_2}=\frac{\lambda_1}{\lambda_2}=\frac{v_1\Delta t}{\lambda_2}
\end{equation}
Waaruit volgt dat:
\begin{equation*}
	\lambda_2=v_2\Delta t
\end{equation*}
Uit driehoek $\Delta ACB$ en $\Delta ADB$ vinden we dat:
\begin{align*}
	\Delta ACB&\Rightarrow sin(\theta_1) = \frac{\lambda_1}{|AB|}\\
	\Delta ADB&\Rightarrow sin(\theta_2) = \frac{\lambda_2}{|AB|}
\end{align*}
Hieruit volgt dat:
\begin{equation*}
	\frac{\lambda_1}{sin(\theta_1)}= \frac{\lambda_2}{sin(\theta_2)} \Rightarrow \frac{\lambda_1}{\lambda_2}=\frac{sin(\theta_1)}{sin(\theta_2)}
\end{equation*}
Met deze informatie kunnen we de inzichten uit vergelijking \ref{eq:verbandSnellius} toepassen om tot het volgende te komen:
\begin{equation*}
	\frac{\lambda_1}{\lambda_2}=\frac{sin(\theta_1)}{sin(\theta_2)}=\frac{v_1}{v_2}= \frac{n_2}{n_1}
\end{equation*}
Na herordening krijgen we de wet van Snellius:
\begin{equation}
	n_1sin(\theta_1)=m_2sin(\theta_2)
\end{equation}
\newpage
\section{Afleiding 29}
\textbf{Regenboog}\\
We bespreken eerst de chromatische dispersie van wit licht. Zoals te zien is in figuur \ref{fig:prisma}, zal bij het breken van licht een splitsing van de kleuren gebeuren. De volgorde van de gesplitste kleuren is niet willekeurig, als de golflengte van licht korter wordt zal de brekingsindex toe nemen zoals te zien is op figuur \ref{fig:chromatische-dispersie}. 
\begin{figure}[!h]
	\centering
	\begin{subfigure}{.5\textwidth}
		\centering
		\includegraphics[width=1\linewidth]{Prisma}
		\caption[Prisma]{Prisma}
		\label{fig:prisma}
	\end{subfigure}%
	\begin{subfigure}{.5\textwidth}
		\centering
		\includegraphics[width=1\linewidth]{"Chromatische dispersie"}
		\caption[Chromatische dispersie]{Chromatische dispersie}
		\label{fig:chromatische-dispersie}
	\end{subfigure}
	\caption{Dispersie}
	\label{fig:ChromDisp}
\end{figure}\\
Door bovenstaand concept toe te passen op een regendruppel die door de lucht valt krijgen we figuur \ref{fig:regendruppelregenboog}.
\begin{figure}[!h]
	\centering
	\includegraphics[width=0.5\linewidth]{RegendruppelRegenboog}
	\caption[Regenboog druppel]{Analyse van één druppel}
	\label{fig:regendruppelregenboog}
\end{figure}
\newpage
Als we enkel rekening houden met één enkele druppel zien we dat de volgorde van kleuren verkeerd is voor een normale regenboog. Daarom moeten we ook rekening houden met het feit dat er meerdere regendruppels nodig zijn om een regenboog te vormen (zie figuur \ref{fig:regenboog}).
\begin{figure}[!h]
	\centering
	\includegraphics[width=0.5\linewidth]{Regenboog}
	\caption[Meerdere druppels]{Er zijn meerdere druppels nodig om een regenboog te vormen}
	\label{fig:regenboog}
\end{figure}\\
Tevens is het mogelijk om een tweede regenboog waar te nemen boven de eerste. Deze noemt men de secundaire regenboog. Ze is minder fel in vergelijking met de eerste doordat er meer interne reflecties gebeuren voor dat het licht vertrekt naar de observeerde persoon.
\begin{figure}[!h]
	\centering
	\includegraphics[width=0.5\linewidth]{SecundaireRegenboog}
	\caption[Secundaire regenboog]{Secundaire regenboog}
	\label{fig:secundaireregenboog}
\end{figure}

\newpage
\section{Afleiding 30}
\label{sec:young}
\textbf{Experiment van Young}\\
Het experiment van Young was belangrijk om het golfkarakter van licht aan te tonen. Concreet werd aangetoond dat lichtgolven met elkaar kunnen interfereren zoals geluidsgolven. Het experiment bestaat uit twee spleten die dicht bij elkaar geplaatst worden, zoals te zien is in figuur \ref{fig:young}. Hierbij is het belangrijk dat de spleten klein genoeg zijn (grote orde $a\approx\lambda$), anders zal er geen buiging van het licht plaats vinden zoals te zien is in figuur \ref{fig:buiginggroot} en \ref{fig:buigingklein}. We merken op dat het golfpatroon op figuur \ref{fig:young} enkel theoretisch is, de reden hiervoor staat in afleiding \ref{sec:RealiteitBuiging}.
\begin{figure}[!h]
	\centering
	\includegraphics[width=0.5\linewidth]{young}
	\caption[Experiment Young]{Experiment Young}
	\label{fig:young}
\end{figure}
\begin{figure}[!h]
	\centering
	\begin{subfigure}{.5\textwidth}
		\centering
		\includegraphics[width=0.7\linewidth]{BuigingGroot}
		\caption[Buiging grote doorlating]{Buiging door groot gat}
		\label{fig:buiginggroot}
	\end{subfigure}%
	\begin{subfigure}{.5\textwidth}
		\centering
		\includegraphics[width=0.9\linewidth]{BuigingKlein}
		\caption[Buiging kleine doorlating]{Buiging door klein gat}
		\label{fig:buigingklein}
	\end{subfigure}
	\caption{Spleten}
	\label{fig:Spleten}
\end{figure}
\newpage
Uit het experiment van Young kunnen we ook de golflengte afleiden van de invallende lichtgolven. Hiervoor kijken we naar figuur \ref{fig:lambdazoeken}:
\begin{figure}[!h]
	\centering
	\includegraphics[width=0.7\linewidth]{LambdaZoeken}
	\caption[Berekening intensiteit]{Intensiteit}
	\label{fig:lambdazoeken}
\end{figure}\\
We kunnen $\Delta L$ schrijven als $L_2-L_1$ en de faseverschuiving als $k\Delta L$. Verder maken we ook nog de veronderstelling dat $L>>d$ waardoor we kunnen zeggen dat $\Delta L \approx dsin(\theta)$. Verder zullen we spreken over constructieve interferentie als voldaan is aan $k\Delta L = 2n\cdot\pi$ met $n\in\mathbb{N}$. Destructieve interferentie doet zich voor als $k\Delta L = (2n+1)\cdot\pi$ met $n\in\mathbb{N}$. Door deze veronderstellingen toe te passen krijgen we:
\begin{align*}
	\text{Constructieve interferentie}\Rightarrow&\frac{2\pi}{\lambda}dsin(\theta) = 2n\pi\\
	\text{Destructieve interferentie}\Rightarrow&\frac{2\pi}{\lambda}dsin(\theta) = (2n+1)\pi
\end{align*}
Als we nu ons eerste maximum zoeken door $n = 1$ te kiezen, krijgen we twee uitdrukkingen:
\begin{align}
	\frac{2\pi}{\lambda}dsin(\theta) &= 2\pi\\
	\label{eq:eerstemax}
	(1)\Rightarrow sin(\theta) &= \frac{\lambda}{d}\\
	\label{eq:eerstemax1}
	(2)\Rightarrow sin(\theta) &\approx tan(\theta) = \frac{H}{L}
\end{align}
Door vergelijking \ref{eq:eerstemax} en \ref{eq:eerstemax1} te combineren krijgen we:
\begin{equation}
	\label{eq:lambdadingen}
	\lambda=\frac{dH}{L}
\end{equation}
\newpage
\section{Afleiding 31}
\label{sec:intensiteitWrong}
\textbf{Intensiteits-profiel}\\
Om een idee te krijgen over de intensiteit van licht bij het experiment van Young, kijken we eerst naar figuur \ref{fig:intensiteitsprofiel}:
\begin{figure}[!h]
	\centering
	\includegraphics[width=0.7\linewidth]{IntensiteitsProfiel}
	\caption[Intensiteits-profiel]{Intensiteits-profiel}
	\label{fig:intensiteitsprofiel}
\end{figure}\\
Het bewijs dat deze figuur klopt volgt uit figuur \ref{fig:lambdazoeken}. We weten nu ook dat licht een golf is, dus kunnen we de elektrische golf vergelijking opschrijven:
\begin{align*}
	E_1&=E_0sin(\omega t)\\
	E_2&=E_0sin(\omega t+\delta)
\end{align*}
Het totaal wordt dan:
\begin{align*}
	E_{\text{Totaal}}&=E_0sin(\omega t)+E_0sin(\omega t+\delta)\\
	&=2E_0sin(\omega t +\frac{\delta}{2})cos(\frac{\delta}{2})
\end{align*}
We maken nu gebruik van het feit dat $\delta = k\Delta L = kdsin(\theta)\approx k\frac{kdH}{L}$
\begin{equation*}
	=2E_0sin(\omega t +\frac{\delta}{2})cos(k\frac{dH}{2L})
\end{equation*}
Nu we de totale golfvergelijking hebben gevonden, kunnen we gebruik maken van de Poynting vector om de intensiteit te bepalen
\begin{equation*}
	I=\bar{S}=\frac{E^2_0}{2\mu_0c}=\frac{4E^2_0}{2\mu_0c}cos^2(k\frac{dH}{2L})
\end{equation*}
\begin{equation*}
	I = I_0cos^2(k\frac{dH}{2L})
\end{equation*}
Het eerste maximum kunnen we vinden door $k\frac{dH}{2L}$ gelijk te stellen aan $\pi$. Door in te vullen in de intensiteits-vergelijking, krijgen we dat:
\begin{equation*}
	\lambda=\frac{dH}{L}
\end{equation*}
Merk op dat deze vergelijking gelijk is aan vergelijking \ref{eq:lambdadingen}. We kunnen analoog werken om de minima te vinden.
\newpage
\section{Afleiding 32}
\label{sec:RealiteitBuiging}
\textbf{Buiging door één opening}\\
\begin{figure}[!h]
	\centering
	\includegraphics[width=0.5\linewidth]{BuigingDoor1Opening}
	\caption[Buiging door 1 opening]{Buiging door één opening}
	\label{fig:buigingdoor1opening}
\end{figure}
Bij afleiding \ref{sec:young} en \ref{sec:intensiteitWrong} is een verkeerde aanname gemaakt. Er is verondersteld dat licht niet zal interfereren als het door één enkele spleet gaat, dit is echter niet het geval zoals aangetoond zal worden in de volgende afleiding. Om de realistische intensiteit te weten te komen op figuur \ref{fig:buigingdoor1opening}, kijken we naar de volgende vier figuren (\ref{fig:buiging1}, \ref{fig:buiging2}, \ref{fig:buiging3}, \ref{fig:buiging4}). 
\begin{figure}[!h]
	\centering
	\includegraphics[width=0.4\linewidth]{BuigingGroot}
	\caption[Buiging 1]{$Dsin(\theta)=0$}
	\label{fig:buiging1}
\end{figure}\\
Hier is er geen enkele interferentie, we zien één heldere plek in het midden van het scherm waarop we projecteren.
\newpage
\begin{figure}[!h]
	\centering
	\includegraphics[width=0.5\linewidth]{Buiging2}
	\caption[Buiging 2]{$Dsin(\theta)=\lambda$}
	\label{fig:buiging2}
\end{figure}
Hier zal de onderste helft van de lichtstralen destructief interfereren. ($ \frac{DH}{L}=\lambda\Rightarrow H=\frac{\lambda L}{D}$)
\begin{figure}[!h]
	\centering
	\includegraphics[width=0.5\linewidth]{Buiging3}
	\caption[Buiging 3]{$Dsin(\theta)=\frac{3\lambda}{2}$}
	\label{fig:buiging3}
\end{figure}\\
Hier zal de onderste twee derden van de lichtstralen destructief interfereren. ($ \frac{DH}{L}=\frac{3\lambda}{2}\Rightarrow H=\frac{3\lambda L}{2D}$)
\newpage
\begin{figure}[!h]
	\centering
	\includegraphics[width=0.5\linewidth]{Buiging4}
	\caption[Buiging 4]{$Dsin(\theta)=2\lambda$}
	\label{fig:buiging4}
\end{figure}
Hier zal de onderste helft van de lichtstralen destructief interfereren. ($ \frac{DH}{L}=2\lambda\Rightarrow H=\frac{2\lambda L}{D}$)
\newline
We zien dus dat de intensiteit sterk af neemt als we $\theta$ toe laten nemen. Dit zorgt er ultiem voor dat de resultaten van het experiment van Young er uit zien als op figuur \ref{fig:intensiteitsprofiel-1}. We zien duidelijk dat de pieken niet gelijk blijven, maar verder afnemen in amplitude naarmate we $\theta$ laten toe nemen. 
\begin{figure}[!h]
	\centering
	\includegraphics[width=0.7\linewidth]{"IntensiteitsProfiel (1)"}
	\caption[Intensiteitsprofiel]{Secundair intensiteitsprofiel}
	\label{fig:intensiteitsprofiel-1}
\end{figure}
\newpage
\section{Afleiding 33}
\textbf{Optische resolutie}
Wanneer twee objecten dicht bij elkaar liggen kunnen we ze niet onderscheiden door buiging. Neem de opstelling in figuur \ref{fig:optischeresolutie}:
\begin{figure}[h]
	\centering
	\includegraphics[width=0.7\linewidth]{OptischeResolutie}
	\caption[Optische resolutie]{Optische resolutie}
	\label{fig:optischeresolutie}
\end{figure}\\
We stellen dat we twee objecten kunnen onderscheiden als voldaan is aan (waarbij D de diameter is van de lens waardoor de objecten bekeken worden):
\begin{equation*}
	\theta\approx\sin(\theta)=\frac{\lambda}{D}
\end{equation*}
In de praktijk vermenigvuldigen we het rechter-lid nog met \num{1.22} om rekening te houden met de optische eigenschappen van het menselijk oog. Concreet krijgen we dus volgende vergelijking:
\begin{equation*}
	\frac{H}{L}= \frac{1.22\cdot \lambda}{D}
\end{equation*}
Als aan bovenstaande voorwaarde voldaan is, zijn we in staat om twee objecten van elkaar ten onderscheiden. We kunnen dus niets dichter dan $\frac{1.22\cdot\lambda L}{D}$ zien. 
\end{document}
